\chapter{Using the system}
\label{chapter.systemuse}
\svnrev{r1936}

\scalaris{} can be used with one of the provided command line interfaces or
by using one of the APIs in a custom program. The following sections will
describe the APIs in general, each API in more detail and the use of our
command line interfaces.

\section{Application Programming Interfaces (APIs)}
\label{chapter.systemuse.apis}

Currently we offer the following APIs:
\begin{itemize}
  \item an \emph{Erlang API} running on the node \scalaris{} is run\\
        (functions can be called using remote connections with distributed
        Erlang)
  \item a \emph{Java API} using Erlang's \code{JInterface} library\\
        (connections are established using distributed Erlang)
  \item a generic \emph{JSON API}\\
        (offered by an integrated HTTP server running on each \scalaris{} node)
  \item a \emph{Python API} for Python >= 2.6 using JSON to talk to \scalaris{}. 
  \item a \emph{Ruby API} for Ruby >= 1.8 using JSON to talk to \scalaris{}. 
\end{itemize}

Each API contains methods for accessing functions from the three layers
\scalaris{} is composed of.
Table~\ref{api.layers} shows the modules and classes of Erlang, Java, Python and Ruby
and their mapping to these layers. Details about the supported operations and
how to access them in each of the APIs are provided in
Section~\sieheref{sec:apis.ops}. A more detailed discussion about the generic
JSON API including examples of JSON calls is shown in
Section~\sieheref{sec.api.json}.

\begin{table}
  \centering
    \begin{tabular}{p{4cm}lll}
    \toprule
                      & Erlang                 & Java                         & Python / Ruby              \\
                      & \footnotesize{module}  & \footnotesize{class in \code{de.zib.scalaris}}%
                                                                              & \footnotesize{class in module \code{scalaris}} \\
    \midrule
    Transaction Layer & \code{api_tx}          & \code{Transaction},          & \code{Transaction},        \\
                      &                        & \code{TransactionSingleOp}   & \code{TransactionSingleOp} \\
    %\cmidrule(lr){2-4}
                      & \code{api_pubsub}      & \code{PubSub}                & \code{PubSub}              \\
    \cmidrule(lr){1-4}
    Replication Layer & \code{api_rdht}        & \code{ReplicatedDHT}         & \code{ReplicatedDHT}       \\
    \cmidrule(lr){1-4}
    P2P Layer         & \code{api_dht}         &                              &                            \\
    %\cmidrule(lr){2-4}
                      & \code{api_dht_raw}     &                              &                            \\
                      & \code{api_vm}          & \code{ScalarisVM}            &                            \\
                      & \code{api_monitor}     &                              &                            \\
    \bottomrule
    \end{tabular}
    \caption{Layered API structure}
    \label{api.layers}
\end{table}

\subsection{Supported Types}

Different programming languages have different types. In order for our APIs
to be compatible with each other, only a subset of the available types is
officially supported.

\emph{Keys} are always strings. In order to avoid problems with different
encodings on different systems, we suggest to only use ASCII characters.

For \emph{values} we distinguish between \emph{native}, \emph{composite}
and \emph{custom} types.

\emph{Native} types are
\begin{itemize}
  \item boolean values
  \item integer numbers
  \item floating point numbers
  \item strings and
  \item binary objects (a number of bytes).
\end{itemize}

\emph{Composite} types are
\begin{itemize}
  \item lists of native types \emph{(except binary objects!)}
  \item JavaScript Object Notation (JSON)\footnote{see \url{http://json.org/}}
\end{itemize}

\emph{Custom} types include any Erlang term not covered by the previous types.
Special care needs to be taken using custom types as they may not be accessible
through every API or may be misinterpreted by an API. The use of them is
discouraged.

Table~\ref{api.supported_types} shows the mapping of supported types to the
language-specific types of each API.

\begin{table}
  \centering
  \begin{threeparttable}[b]
    \begin{tabular}{llllll}
    \toprule
               & Erlang            & Java                         & JSON                            & Python                    & Ruby \\
    \midrule
    boolean    & \code{boolean()}  & \code{bool}, \code{Boolean}  & \code{true}, \code{false}       & \code{True}, \code{False} & \code{true}, \code{false} \\
    integer    & \code{integer()}  & \code{int}, \code{Integer}   & \code{int}                      & \code{int}                & \code{Fixnum}, \\
               &                   & \code{long}, \code{Long}     &                                 &                           & \code{Bignum} \\
               &                   & \code{BigInteger}            &                                 &                           & \\
    float      & \code{float()}    & \code{double}, \code{Double} & \code{int frac}                 & \code{float}              & \code{Float} \\
               &                   &                              & \code{int exp}                  &                           & \\
               &                   &                              & \code{int frac exp}             &                           & \\
    string     & \code{string()}   & \code{String}                & \code{string}                   & \code{str}                & \code{String} \\
    binary     & \code{binary()}   & \code{byte[]}                & \code{string}                   & \code{bytearray}          & \code{String} \\
               &                   &                              & \footnotesize{(base64-encoded)} &                           & \\
    list(type) & \code{[type()]}   & \code{List<Object>}          & \code{array}                    & \code{list}               & \code{Array} \\
    JSON       & \code{json_obj()}\tnote{*} & \code{Map<String, Object>} & \code{object}            & \code{dict}               & \code{Hash}\\
    custom     & \code{any()}      & \code{OtpErlangObject}       & \emph{/}                        & \emph{/}                  & \emph{/} \\
    \bottomrule
    \end{tabular}
    \begin{tablenotes}
      \item[*] ~\vspace{-1.5em}%
\begin{lstlisting}[language=erlang]
json_obj() :: {struct, [Key::atom() | string(), Value::json_val()]}
json_val() :: string() | number() | json_obj() | {array, [any()]} | true | false | null
\end{lstlisting}
    \end{tablenotes}
    \caption{Types supported by the \scalaris{} APIs}
    \label{api.supported_types}
  \end{threeparttable}
\end{table}

\subsection{Supported Operations}
\label{sec:apis.ops}

Most operations are available to all APIs, but some (especially convenience
methods) are API- or language-specific. The following paragraphs
provide a brief overview of what is available to which API. For a full
reference, see the documentation of the specific API.

\subsubsection{Transaction Layer}

\paragraph{Read}
Reads the value stored at a given key using quorum read.

\begin{tabular}{lp{14cm}}
Erlang  & \code{api_tx:read(Key)}\\
Java:   & \code{TransactionSingleOp.read(Key)}\\
JSON:   & \code{read(Key)}\\
Python: & \code{TransactionSingleOp.read(Key)}\\
Ruby:   & \code{TransactionSingleOp.read(Key)}
\end{tabular}

\paragraph{Write}
Writes a value to a given key.

\begin{tabular}{lp{14cm}}
Erlang  & \code{api_tx:write(Key, Value)}\\
Java:   & \code{TransactionSingleOp.write(Key, Value)}\\
JSON:   & \code{write(Key, Value)}\\
Python: & \code{TransactionSingleOp.write(Key, Value)}\\
Ruby:   & \code{TransactionSingleOp.write(Key, Value)}
\end{tabular}

\paragraph{``Add to'' \& ``Delete from'' List Operations}
For the list stored at a given key, first add all elements from a given list,
then remove all elements from a second given list.

\begin{tabular}{lp{14cm}}
Erlang  & \code{api_tx:add_del_on_list(Key, ToAddList, ToRemoveList)}\\
Java:   & \code{TransactionSingleOp.addDelOnList(Key, ToAddList, ToRemoveList)}\\
JSON:   & \code{add_del_on_list(Key, ToAddList, ToRemoveList)}\\
Python: & \code{TransactionSingleOp.add_del_on_list(Key, ToAddList, ToRemoveList)}\\
Ruby:   & \code{TransactionSingleOp.add_del_on_list(Key, ToAddList, ToRemoveList)}
\end{tabular}

\paragraph{Add to a number}
Adds a given number to the number stored at a given key.

\begin{tabular}{lp{14cm}}
Erlang  & \code{api_tx:add_on_nr(Key, ToAddNumber)}\\
Java:   & \code{TransactionSingleOp.addOnNr(Key, ToAddNumber)}\\
JSON:   & \code{add_on_nr(Key, ToAddList, ToAddNumber)}\\
Python: & \code{TransactionSingleOp.add_on_nr(Key, ToAddNumber)}\\
Ruby:   & \code{TransactionSingleOp.add_on_nr(Key, ToAddNumber)}
\end{tabular}

\paragraph{Atomic Test and Set}
Writes the given (new) value to a key if the current value is equal to the
given old value.

\begin{tabular}{lp{14cm}}
Erlang  & \code{api_tx:test_and_set(Key, OldValue, NewValue)}\\
Java:   & \code{TransactionSingleOp.testAndSet(Key, OldValue, NewValue)}\\
JSON:   & \code{add_on_nr(Key, OldValue, NewValue)}\\
Python: & \code{TransactionSingleOp.test_and_set(Key, OldValue, NewValue)}\\
Ruby:   & \code{TransactionSingleOp.test_and_set(Key, OldValue, NewValue)}
\end{tabular}

\paragraph{Bulk Operations}
Executes multiple requests, i.e. operations, where each of them will be
committed.

\emph{Collecting requests and executing all of them in a single call yields
better performance than executing all on their own.}

\begin{tabular}{lp{14cm}}
Erlang  & \code{api_tx:req_list_commit_each(RequestList)}\\
Java:   & \code{TransactionSingleOp.req_list(RequestList)}\\
JSON:   & \code{req_list_commit_each(RequestList)}\\
Python: & \code{TransactionSingleOp.req_list(RequestList)}\\
Ruby:   & \code{TransactionSingleOp.req_list(RequestList)}
\end{tabular}

\subsubsection{Transaction Layer (with TLog)}

\paragraph{Read (with TLog)}
Reads the value stored at a given key using quorum read as an additional part
of a previous transaction or for starting a new one \emph{(no auto-commit!)}.

\begin{tabular}{lp{14cm}}
Erlang  & \code{api_tx:read(TLog, Key)}\\
Java:   & \code{Transaction.read(Key)}\\
JSON:   & \code{n/a} - use \code{req_list}\\
Python: & \code{Transaction.read(Key)}\\
Ruby:   & \code{Transaction.read(Key)}
\end{tabular}

\paragraph{Write (with TLog)}
Writes a value to a given key as an additional part
of a previous transaction or for starting a new one \emph{(no auto-commit!)}.

\begin{tabular}{lp{14cm}}
Erlang  & \code{api_tx:write(TLog, Key, Value)}\\
Java:   & \code{Transaction.write(Key, Value)}\\
JSON:   & \code{n/a} - use \code{req_list}\\
Python: & \code{Transaction.write(Key, Value)}\\
Ruby:   & \code{Transaction.write(Key, Value)}
\end{tabular}

\paragraph{``Add to'' \& ``Delete from'' List Operations (with TLog)}
For the list stored at a given key, first add all elements from a given list,
then remove all elements from a second given list as an additional part
of a previous transaction or for starting a new one \emph{(no auto-commit!)}.

\begin{tabular}{lp{14cm}}
Erlang  & \code{api_tx:add_del_on_list(TLog, Key, ToAddList, ToRemoveList)}\\
Java:   & \code{Transaction.addDelOnList(Key, ToAddList, ToRemoveList)}\\
JSON:   & \code{n/a} - use \code{req_list}\\
Python: & \code{Transaction.add_del_on_list(Key, ToAddList, ToRemoveList)}\\
Ruby:   & \code{Transaction.add_del_on_list(Key, ToAddList, ToRemoveList)}
\end{tabular}

\paragraph{Add to a number (with TLog)}
Adds a given number to the number stored at a given key as an additional part
of a previous transaction or for starting a new one \emph{(no auto-commit!)}.

\begin{tabular}{lp{14cm}}
Erlang  & \code{api_tx:add_on_nr(TLog, Key, ToAddNumber)}\\
Java:   & \code{Transaction.addOnNr(Key, ToAddNumber)}\\
JSON:   & \code{n/a} - use \code{req_list}\\
Python: & \code{Transaction.add_on_nr(Key, ToAddNumber)}\\
Ruby:   & \code{Transaction.add_on_nr(Key, ToAddNumber)}
\end{tabular}

\paragraph{Atomic Test and Set (with TLog)}
Writes the given (new) value to a key if the current value is equal to the
given old value as an additional part
of a previous transaction or for starting a new one \emph{(no auto-commit!)}.

\begin{tabular}{lp{14cm}}
Erlang  & \code{api_tx:test_and_set(TLog, Key, OldValue, NewValue)}\\
Java:   & \code{Transaction.testAndSet(Key, OldValue, NewValue)}\\
JSON:   & \code{test_and_set(Key, OldValue, NewValue)}\\
Python: & \code{Transaction.test_and_set(Key, OldValue, NewValue)}\\
Ruby:   & \code{Transaction.test_and_set(Key, OldValue, NewValue)}
\end{tabular}

\paragraph{Bulk Operations (with TLog)}
Executes multiple requests, i.e. operations, as an additional part
of a previous transaction or for starting a new one \emph{(no auto-commit!)}.
Only one \code{commit} request is allowed per call!

\emph{Collecting requests and executing all of them in a single call yields
better performance than executing all on their own.}

\begin{tabular}{lp{14cm}}
Erlang  & \code{api_tx:req_list(RequestList)}, \code{api_tx:req_list(TLog, RequestList)}\\
Java:   & \code{Transaction.req_list(RequestList)}\\
JSON:   & \code{req_list(RequestList)}, \code{req_list(TLog, RequestList)}\\
Python: & \code{Transaction.req_list(RequestList)}\\
Ruby:   & \code{Transaction.req_list(RequestList)}
\end{tabular}

\subsubsection{Transaction Layer (Pub/Sub)}
Scalaris implements a simple Publish/Subscribe system. Subscribers can
subscribe URLs for some topic. If an event is published to that topic, a
JSON-RPC is send to that URL, i.e. a JSON object of the following form:
\begin{lstlisting}[language=java]
{
 "method": "notify",
 "params": [<topic>, <content>],
 "id"    : <number>
}
\end{lstlisting}

\paragraph{Publish}
Publishes an event under a given topic.

\begin{tabular}{lp{14cm}}
Erlang  & \code{api_pubsub:publish(Topic, Content)}\\
Java:   & \code{PubSub.publish(Topic, Content)}\\
JSON:   & \code{publish(Topic, Content)}\\
Python: & \code{PubSub.publish(Topic, Content)}\\
Ruby:   & \code{PubSub.publish(Topic, Content)}
\end{tabular}

\paragraph{Subscribe}
Subscribes a URL for a topic.

\begin{tabular}{lp{14cm}}
Erlang  & \code{api_pubsub:subscribe(Topic, URL)}\\
Java:   & \code{PubSub.subscribe(Topic, URL)}\\
JSON:   & \code{subscribe(Topic, URL)}\\
Python: & \code{PubSub.subscribe(Topic, URL)}\\
Ruby:   & \code{PubSub.subscribe(Topic, URL)}
\end{tabular}

\paragraph{Unsubscribe}
Subscribes a URL from a topic.

\begin{tabular}{lp{14cm}}
Erlang  & \code{api_pubsub:unsubscribe(Topic, URL)}\\
Java:   & \code{PubSub.unsubscribe(Topic, URL)}\\
JSON:   & \code{unsubscribe(Topic, URL)}\\
Python: & \code{PubSub.unsubscribe(Topic, URL)}\\
Ruby:   & \code{PubSub.unsubscribe(Topic, URL)}
\end{tabular}

\paragraph{Get Subscribers}
Gets a list of subscribed URLs for a given topic.

\begin{tabular}{lp{14cm}}
Erlang  & \code{api_pubsub:get_subscribers(Topic)}\\
Java:   & \code{PubSub.getSubscribers(Topic)}\\
JSON:   & \code{get_subscribers(Topic)}\\
Python: & \code{PubSub.get_subscribers(Topic)}\\
Ruby:   & \code{PubSub.get_subscribers(Topic)}
\end{tabular}

\subsubsection{Replication Layer}

\paragraph{Delete}
Tries to delete a value at a given key.

\emph{Warning: This can only be done outside the transaction layer and is thus not
absolutely safe. Refer to the following thread on the mailing list:
\url{http://groups.google.com/group/scalaris/browse_thread/thread/ff1d9237e218799}}.

\begin{tabular}{lp{14cm}}
Erlang  & \code{api_rdht:delete(Key)}, \code{api_rdht:delete(Key, Timeout)}\\
Java:   & \code{ReplicatedDHT.delete(Key)}, \code{ReplicatedDHT.delete(Key, Timeout)}\\
JSON:   & \code{delete(Key)}, \code{delete(Key, Timeout)}\\
Python: & \code{ReplicatedDHT.delete(Key)}, \code{ReplicatedDHT.delete(Key, Timeout)}\\
Ruby:   & \code{ReplicatedDHT.delete(Key)}, \code{ReplicatedDHT.delete(Key, Timeout)}
\end{tabular}

\paragraph{Get Replica Keys}
Gets the (hashed) keys used for the replicas of a given (user) key
(ref. Section~\nameref{sec:apis.ops.p2player}).

\begin{tabular}{lp{14cm}}
Erlang  & \code{api_rdht:get_replica_keys(Key)}\\
Java:   & \code{n/a}\\
JSON:   & \code{n/a}\\
Python: & \code{n/a}\\
Ruby:   & \code{n/a}
\end{tabular}

\subsubsection{P2P Layer}
\label{sec:apis.ops.p2player}

\paragraph{Hash Key}
Generates the hash of a given (user) key.

\begin{tabular}{lp{14cm}}
Erlang  & \code{api_dht:hash_key(Key)}\\
Java:   & \code{n/a}\\
JSON:   & \code{n/a}\\
Python: & \code{n/a}\\
Ruby:   & \code{n/a}
\end{tabular}

\paragraph{Get Replica Keys}
Gets the (hashed) keys used for the replicas of a given (hashed) key.

\begin{tabular}{lp{14cm}}
Erlang  & \code{api_dht_raw:get_replica_keys(HashedKey)}\\
Java:   & \code{n/a}\\
JSON:   & \code{n/a}\\
Python: & \code{n/a}\\
Ruby:   & \code{n/a}
\end{tabular}

\paragraph{Range Read}
Reads all Key-Value pairs in a given range of (hashed) keys.

\begin{tabular}{lp{14cm}}
Erlang  & \code{api_dht_raw:range_read(StartHashedKey, EndHashedKey)}\\
Java:   & \code{n/a}\\
JSON:   & \code{n/a}\\
Python: & \code{n/a}\\
Ruby:   & \code{n/a}
\end{tabular}

\subsubsection{P2P Layer (VM Management)}

\paragraph{Get Scalaris Version}
Gets the version of \scalaris{} running in the requested Erlang VM.

\begin{tabular}{lp{14cm}}
Erlang  & \code{api_vm:get_version()}\\
Java:   & \code{ScalarisVM.getVersion()}\\
JSON:   & \code{n/a}\\
Python: & \code{n/a}\\
Ruby:   & \code{n/a}
\end{tabular}

\paragraph{Get Node Info}
Gets various information about the requested Erlang VM and the running
\scalaris{} code, e.g. \scalaris{} version, erlang version, memory use, uptime.

\begin{tabular}{lp{14cm}}
Erlang  & \code{api_vm:get_info()}\\
Java:   & \code{ScalarisVM.getInfo()}\\
JSON:   & \code{n/a}\\
Python: & \code{n/a}\\
Ruby:   & \code{n/a}
\end{tabular}

\paragraph{Get Information about Different VMs}
Get connection info about other Erlang VMs running \scalaris{} nodes.
Note: This info is provided by the \erlmodule{cyclon} service built into
\scalaris{}.

\begin{tabular}{lp{14cm}}
Erlang  & \code{api_vm:get_other_vms(MaxVMs)}\\
Java:   & \code{ScalarisVM.getOtherVMs(MaxVMs)}\\
JSON:   & \code{n/a}\\
Python: & \code{n/a}\\
Ruby:   & \code{n/a}
\end{tabular}

\paragraph{Get Number of Scalaris Nodes in the VM}
Gets the number of Scalaris nodes running inside the Erlang VM.

\begin{tabular}{lp{14cm}}
Erlang  & \code{api_vm:number_of_nodes()}\\
Java:   & \code{ScalarisVM.getNumberOfNodes()}\\
JSON:   & \code{n/a}\\
Python: & \code{n/a}\\
Ruby:   & \code{n/a}
\end{tabular}

\paragraph{Get Scalaris Nodes}
Gets a list of Scalaris nodes running inside the Erlang VM.

\begin{tabular}{lp{14cm}}
Erlang  & \code{api_vm:get_nodes()}\\
Java:   & \code{ScalarisVM.getNodes()}\\
JSON:   & \code{n/a}\\
Python: & \code{n/a}\\
Ruby:   & \code{n/a}
\end{tabular}

\paragraph{Add Scalaris Nodes}
Starts additional Scalaris nodes inside the Erlang VM.

\begin{tabular}{lp{14cm}}
Erlang  & \code{api_vm:add_nodes(Number)}\\
Java:   & \code{ScalarisVM.addNodes(Number)}\\
JSON:   & \code{n/a}\\
Python: & \code{n/a}\\
Ruby:   & \code{n/a}
\end{tabular}

\paragraph{Shutdown Scalaris Nodes}
Gracefully kill some Scalaris nodes inside the Erlang VM. This will first move
the data from the nodes to other nodes and then shut them down.

\begin{tabular}{lp{14cm}}
Erlang  & \code{api_vm:shutdown_node(Name)},\newline
          \code{api_vm:shutdown_nodes(Count)}, \code{api_vm:shutdown_nodes_by_name(Names)}\\
Java:   & \code{ScalarisVM.shutdownNode(Name)},\newline
          \code{ScalarisVM.shutdownNodes(Number)}, \code{ScalarisVM.shutdownNodesByName(Names)}\\
JSON:   & \code{n/a}\\
Python: & \code{n/a}\\
Ruby:   & \code{n/a}
\end{tabular}

\paragraph{Kill Scalaris Nodes}
Immediately kills some Scalaris nodes inside the Erlang VM.

\begin{tabular}{lp{14cm}}
Erlang  & \code{api_vm:kill_node(Name)},\newline
          \code{api_vm:kill_nodes(Count)}, \code{api_vm:kill_nodes_by_name(Names)}\\
Java:   & \code{ScalarisVM.killNode(Name)},\newline
          \code{ScalarisVM.killNodes(Number)}, \code{ScalarisVM.killNodesByName(Names)}\\
JSON:   & \code{n/a}\\
Python: & \code{n/a}\\
Ruby:   & \code{n/a}
\end{tabular}

\paragraph{Shutdown the Erlang VM}
Gracefully shuts down all \scalaris{} nodes in the Erlang VM and then exits.

\begin{tabular}{lp{14cm}}
Erlang  & \code{api_vm:shutdown_vm()}\\
Java:   & \code{ScalarisVM.shutdownVM()}\\
JSON:   & \code{n/a}\\
Python: & \code{n/a}\\
Ruby:   & \code{n/a}
\end{tabular}

\paragraph{Kill the Erlang VM}
Immediately kills all \scalaris{} nodes in the Erlang VM and then exits.

\begin{tabular}{lp{14cm}}
Erlang  & \code{api_vm:kill_vm()}\\
Java:   & \code{ScalarisVM.killVM()}\\
JSON:   & \code{n/a}\\
Python: & \code{n/a}\\
Ruby:   & \code{n/a}
\end{tabular}

\subsubsection{Convenience Methods / Classes}

\paragraph{Connection Pool}
Implements a thread-safe pool of connections to Scalaris instances. Can be
instantiated with a fixed maximum number of connections. Connections are either taken from
a pool of available connections or are created on demand. If finished, a connection can be
put back into the pool.

\begin{tabular}{ll}
Erlang  & \code{n/a}\\
Java:   & \code{ConnectionPool}\\
JSON:   & \code{n/a}\\
Python: & \code{ConnectionPool}\\
Ruby:   & \code{n/a}
\end{tabular}

\paragraph{Connection Policies}
Defines policies on how to select a node to connect to from a set of
possible nodes and whether and how to automatically re-connect.

\begin{tabular}{ll}
Erlang  & \code{n/a}\\
Java:   & \code{ConnectionPolicy}\\
JSON:   & \code{n/a}\\
Python: & \code{n/a}\\
Ruby:   & \code{n/a}
\end{tabular}

\subsection{JSON API}
\label{sec.api.json}

\scalaris{} supports a JSON API for transactions. To minimize the necessary
round trips between a client and \scalaris{}, it uses request lists, which
contain all requests that can be done in parallel. The request list is then
send to a \scalaris{} node with a POST message. The result contains
a list of the results of the requests and - in case of a transaction - a
TransLog. To add further
requests to the transaction, the TransLog and another list of requests may
be send to \scalaris{}. This process may be repeated as often as necessary.
To finish the transaction, the request list can contain a 'commit' request
as the last element, which triggers the validation phase of the transaction
processing.
Request lists are also supported for single read/write operations, i.e.
every single operation is committed on its own. 

The JSON-API can be accessed via the \scalaris{}-Web-Server running on port
8000 by default and the page \code{jsonrpc.yaws} (For example at:
\url{http://localhost:8000/jsonrpc.yaws}).
Requests are issued by sending a JSON object with header
\code{"Content-type"="application/json"} to this URL.
The result will then be returned as a JSON object with the same content type.
The following table shows how both objects look like:

\begin{tabular}{p{0.45\textwidth}cp{0.45\textwidth}}
\bf Request & & \bf Result \\
\begin{lstlisting}[language=java]
{
  "jsonrpc": "2.0",
  "method" : "<method>",
  "params" : [<params>],
  "id"     : <number>
}
\end{lstlisting}
& &
\begin{lstlisting}[language=java]
{
  "result" : <result_object>,
  "id"     : <number>
}
\end{lstlisting}
\end{tabular}

The \code{id} in the request can be an arbitrary number which identifies the
request and is returned in the result.
The following operations (shown as \code{<method>(<params>)}) are currently
supported (the given result is the \code{<result_object>} mentioned above):
\begin{itemize}
  \item \code{nop(Value)} - no operation, result:
\begin{lstlisting}[language=java]
"ok"
\end{lstlisting}
  \item[] \hspace{-1.7em}single operations, e.g. read/write:
  \item \code{req_list_commit_each(<req_list_ce>)} - commit each request in the list, result:
\begin{lstlisting}[language=java]
{[{"status": "ok"} or {"status": "ok", "value": <json_value>} or
  {"status": "fail", "reason": "timeout" or "abort" or "not_found" or
                               "not_a_list" or "not_a_number"} or
  {"status": "fail", "reason": "key_changed", "value": <json_value>}]}
\end{lstlisting}
  \item \code{read(<key>)} - read the value at \code{key}, result:
\begin{lstlisting}[language=java]
{"status": "ok", "value", <json_value>} or
{"status": "fail", "reason": "timeout" or "not_found"}
\end{lstlisting}
  \item \code{write(<key>, <json_value>)} - write \code{value} (inside \code{json_value}) to \code{key}, result:
\begin{lstlisting}[language=java]
{"status": "ok"} or
{"status": "fail", "reason": "timeout" or "abort"}
\end{lstlisting}
  \item \code{add_del_on_list(<key>, ToAdd, ToRemove)} - adding to / removing from a list
  (for the list at \code{key} adds all values in the \code{ToAdd} list and
  then removes all values in the \code{ToRemove} list; if there is no value at
  \code{key}, uses an empty list - both value lists are \code{[<value>]}), result:
\begin{lstlisting}[language=java]
{"status": "ok"} or
{"status": "fail", "reason": "timeout" or "abort" or "not_a_list"}
\end{lstlisting}
  \item \code{add_on_nr(<key>, <value>)} - adding to a number
  (adds \code{value} to the number at \code{key} - both values must be numbers), result:
\begin{lstlisting}[language=java]
{"status": "ok"} or
{"status": "fail", "reason": "timeout" or "abort" or "not_a_number"}
\end{lstlisting}
  \item \code{test_and_set(<key>, OldValue, NewValue)} - atomic test-and-set
  (write \code{NewValue} to \code{key} if the current value is \code{OldValue}
   - both values are \code{<json_value>}), result:
\begin{lstlisting}[language=java]
{"status": "ok"} or
{"status": "fail", "reason": "timeout" or "abort" or "not_found"} or
{"status": "fail", "reason": "key_changed", "value": <json_value>}
\end{lstlisting}
  \item[] \hspace{-1.7em}transactions:
  \item \code{req_list(<req_list>)} - process a list of requests, result:
\begin{lstlisting}[language=java]
{"tlog": <tlog>,
 "results": [{"status": "ok"} or {"status": "ok", "value": <json_value>} or
             {"status": "fail", "reason": "timeout" or "abort" or "not_found" or
                                          "not_a_list" or "not_a_number"} or
             {"status": "fail", "reason": "key_changed", "value": <json_value>}]}
\end{lstlisting}
  \item \code{req_list(<tlog>, <req_list>)} - process a list of requests with a previous translog, result:
\begin{lstlisting}[language=java]
{"tlog": <tlog>,
 "results": [{"status": "ok"} or {"status": "ok", "value": <json_value>} or
             {"status": "fail", "reason": "timeout" or "abort" or "not_found" or
                                          "not_a_list" or "not_a_number"} or
             {"status": "fail", "reason": "key_changed", "value": <json_value>}]}
\end{lstlisting}
  \item[] \hspace{-1.7em}replication layer functions:
  \item \code{delete(<key>)} - delete the value at \code{key}, default timeout 2s, result:
\begin{lstlisting}[language=java]
{"ok": <number>, "results": ["ok" or "locks_set" or "undef"]} or
{"failure": "timeout", "ok": <number>, "results": ["ok" or "locks_set" or "undef"]}
\end{lstlisting}
  \item \code{delete(<key>, Timeout)} - delete the value at \code{key} with a timeout of \code{Timeout} Milliseconds, result:
\begin{lstlisting}[language=java]
{"ok": <number>, "results": ["ok" or "locks_set" or "undef"]} or
{"failure": "timeout", "ok": <number>, "results": ["ok" or "locks_set" or "undef"]}
\end{lstlisting}
  \item[] \hspace{-1.7em}raw DHT functions:
  \item \code{range_read(From, To)} - read a range of (raw) keys, result:
\begin{lstlisting}[language=java]
{"status": "ok" or "timeout",
 "value": [{"key": <key>, "value": <json_value>, "version": <version>}]}
\end{lstlisting}
  \item[] \hspace{-1.7em}publish/subscribe:
  \item \code{publish(Topic, Content)} - publish \code{Content} to \code{Topic} (\code{<key>}), result:
\begin{lstlisting}[language=java]
{"status": "ok"}
\end{lstlisting}
  \item \code{subscribe(Topic, URL)} - subscribe \code{URL} to \code{Topic} (\code{<key>}), result:
\begin{lstlisting}[language=java]
{"status": "ok"} or
{"status": "fail", "reason": "timeout" or "abort"}
\end{lstlisting}
  \item \code{unsubscribe(Topic, URL)} - unsubscribe \code{URL} from \code{Topic} (\code{<key>}), result:
\begin{lstlisting}[language=java]
{"status": "ok"} or
{"status": "fail", "reason": "timeout" or "abort" or "not_found"}
\end{lstlisting}
  \item \code{get_subscribers(Topic)} - get subscribers of \code{Topic} (\code{<key>}), result:
\begin{lstlisting}[language=java]
[<urls>]
\end{lstlisting}
%  \item \code{notify(Topic, Value)} - process a notify which came from a publish, result:
% \begin{lstlisting}[language=java]
% "ok"
% \end{lstlisting}
\end{itemize}

Note:
\begin{lstlisting}[language=java]
<json_value> = {"type": "as_is" or "as_bin", "value": <value>}
<operation> = {"read": <key>} or {"write", {<key>: <json_value>}} or
              {"add_del_on_list": {"key": <key>, "add": [<value>], "del": [<value>]}} or
              {"add_on_nr": {<key>: <value>}} or
              {"test_and_set": {"key": <key>, "old": <json_value>, "new": <json_value>}}
<req_list_ce> = [<operation>]
<req_list> = [<operation> or {"commit", _}]
\end{lstlisting}
The \code{<value>} inside \code{<json_value>} is either a base64-encoded
string representing a binary object (type = \code{"as_bin"}) or the value
itself (type = \code{"as_is"}).

\subsubsection{JSON-Example}

The following example illustrates the message flow:

%note: use tables inside tables instead of multicol for easier listings handling
\begin{longtable}{p{0.97\textwidth}}
\begin{tabular}{p{0.43\textwidth}cp{0.43\textwidth}}
\bf Client & & \hfill\bf \scalaris{} node \\
\end{tabular} \\
%
% request:
\begin{tabular}{p{0.43\textwidth}cp{0.43\textwidth}}
Make a transaction, that sets two keys & $\to$ & \\
\end{tabular}\vspace{-1.5em} \\
%
\begin{tabular}{p{0.78\textwidth}p{0.13\textwidth}}
\vspace{-1.5em}%
\begin{lstlisting}[language=java]
{"jsonrpc": "2.0",
 "method": "req_list",
 "params": [
  [ { "write": { "keyA": {"type": "as_is", "value": "valueA"} } },
    { "write": { "keyB": {"type": "as_is", "value": "valueB"} } },
    { "commit": "" } ]
 ],
 "id": 0
}
\end{lstlisting}
& \\
\end{tabular}\vspace{-1em} \\
%
% result:
\begin{tabular}{p{0.43\textwidth}cp{0.43\textwidth}}
 & $\leftarrow$ & \hfill{}\scalaris{} sends results back \\
\end{tabular}\vspace{-1.5em} \\

\begin{tabular}{p{0.13\textwidth}p{0.78\textwidth}}
& 
\vspace{-1.5em}%
\begin{lstlisting}[language=java]
{"error": null,
 "result": {
  "results": [ {"status": "ok"}, {"status": "ok"}, {"status": "ok"} ],
  "tlog": <TLOG> // this is the translog for further operations!
 },
 "id": 0
}
\end{lstlisting} \\
\end{tabular}\vspace{-1em} \\
%
% request:
\begin{tabular}{p{0.43\textwidth}cp{0.43\textwidth}}
In a second transaction: Read the two keys & $\to$ & \\
\end{tabular}\vspace{-1.5em} \\
%
\begin{tabular}{p{0.78\textwidth}p{0.13\textwidth}}
\vspace{-1.5em}%
\begin{lstlisting}[language=java]
{"jsonrpc": "2.0",
 "method": "req_list",
 "params": [
  [ { "read": "keyA" },
    { "read": "keyB" } ]
 ],
 "id": 0
}
\end{lstlisting}
& \\
\end{tabular}\vspace{-1em} \\
%
% result:
\begin{tabular}{p{0.43\textwidth}cp{0.43\textwidth}}
 & $\leftarrow$ & \hfill{}\scalaris{} sends results back \\
\end{tabular}\vspace{-1.5em} \\

\begin{tabular}{p{0.13\textwidth}p{0.78\textwidth}}
& 
\vspace{-1.5em}%
\begin{lstlisting}[language=java]
{"error": null,
 "result": {
  "results": [
   { "status": "ok", "value": {"type": "as_is", "value": "valueA"} },
   { "status": "ok", "value": {"type": "as_is", "value": "valueB"} }
  ],
  "tlog": <TLOG>
 },
 "id": 0
}
\end{lstlisting} \\
\end{tabular} \\
%
% request:
\begin{tabular}{p{0.43\textwidth}cp{0.43\textwidth}}
Calculate something with the read values and make further requests, here a
write and the commit for the whole transaction. Also include the latest
translog we got from \scalaris{} (named \code{<TLOG>} here). & $\to$ & \\
\end{tabular}\vspace{-1.5em} \\
%
\begin{tabular}{p{0.78\textwidth}p{0.13\textwidth}}
\vspace{-1.5em}%
\begin{lstlisting}[language=java]
{"jsonrpc": "2.0",
 "method": "req_list",
 "params": [
  <TLOG>,
  [ { "write": { "keyA": {"type": "as_is", "value": "valueA2"} } },
    { "commit": "" } ]
 ],
 "id": 0
}
\end{lstlisting}
& \\
\end{tabular}\vspace{-1em} \\
%
% result:
\begin{tabular}{p{0.43\textwidth}cp{0.43\textwidth}}
 & $\leftarrow$ & \hfill{}\scalaris{} sends results back \\
\end{tabular}\vspace{-1.5em} \\

\begin{tabular}{p{0.13\textwidth}p{0.78\textwidth}}
& 
\vspace{-1.5em}%
\begin{lstlisting}[language=java]
{"error": null,
 "result": {
  "results": [ {"status": "ok"}, {"status": "ok"} ],
  "tlog": <TLOG>
 },
 "id": 0
}
\end{lstlisting} \\
\end{tabular} \\
\end{longtable}

Examples of how to use the JSON API are the Python and Ruby API which use JSON
to communicate with \scalaris{}.

%\subsection{Erlang}

\subsection{Java API}

The \code{scalaris.jar} provides a Java command line client as well as a
library for Java programs to access \scalaris{}. The library provides several
classes:

\begin{itemize}
\item \code{TransactionSingleOp} provides methods for reading and writing values.
\item \code{Transaction} provides methods for reading and writing values in transactions.
\item \code{PubSub} provides methods for a simple topic-based pub/sub implementation on top of \scalaris{}.
\item \code{ReplicatedDHT} provides low-level methods for accessing the replicated DHT of \scalaris{}.
\end{itemize}

For details regarding the API we refer the reader to the Javadoc:

\begin{lstlisting}[language=sh]
%> cd java-api
%> ant doc
%> firefox doc/index.html
\end{lstlisting}

\section{Command Line Interfaces}

\subsection{Java command line interface}

As mentioned above, the \code{scalaris.jar} file contains a small command line
interface client. For
convenience, we provide a wrapper script called \code{scalaris} which
sets up the Java environment:

\begin{lstlisting}[language=sh]
%> ./java-api/scalaris --noconfig --help
\end{lstlisting}
\lstinputlisting[language={}]{scalaris-client-java.out}

\code{read}, \code{write} and \code{delete} can be used to read, write
and delete from/to the overlay, respectively. \code{getsubscribers},
\code{publish}, and \code{subscribe} are the PubSub functions. The others
provide debugging and testing functionality.

\begin{lstlisting}[language=]
%> ./java-api/scalaris -write foo bar
write(foo, bar)
%> ./java-api/scalaris -read foo
read(foo) == bar
\end{lstlisting}

Per default, the \code{scalaris} script tries to connect to a management
server at \code{localhost}. You can change the node it connects to (and
further connection properties) by adapting the values defined in
\code{java-api/scalaris.properties}.

\subsection{Python command line interface}

\begin{lstlisting}[language=sh]
%> ./python-api/scalaris_client.py --help
\end{lstlisting}
\lstinputlisting[language={}]{scalaris-client-python.out}

\subsection{Ruby command line interface}

\begin{lstlisting}[language=sh]
%> ../ruby-api/scalaris_client.rb --help
\end{lstlisting}
\lstinputlisting[language={}]{scalaris-client-ruby.out}
